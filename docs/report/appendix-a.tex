% $Id: appendix-a.tex 274 2003-05-07 16:22:37Z mackers $

SightWeaver is a tool for repairing HTML tables so that they comply with
existing accessibility standards. The program imports existing HTML documents,
and attempts to determine accessibility information from the markup. The user
is then presented with the corrected tables and may verify and refine the table
structure and accessibilty information before exporting the document.

Features:

\begin{itemize}

\item Can import XHTML, HTML and CSV files formats. Invalid or badly marked up HTML is supported by using JTidy.

\item Microsoft Word and Excel ``Save-As-HTML'' documents are also explicitly supported.

\item All existing accessibility information in the original document is preserved and re-used.

\item Documents can contain up to 10 tables. Non-table content is preserved and also ``cleaned'' using JTidy.

\item Robust table parser - cell spanning, short rows, thead, etc.

\item The user is given full control over all accessibility information in the tables; including table summaries, captions and cell headers structure, header associations, axes and abbreviations.

\item Output is well-formed, valid XHTML.

\item (Table) output conforms to WCAG and Section 508 web standards.

\item Program conforms to ATAG.

\end{itemize}

\section*{Running the Program}

SightWeaver is a java application and therefore must have the Java Runtime
Environment (JRE) installed in order to use. The JRE can be downloaded from Sun
Microsystem's website\footnote{http://www.sun.com/} . Version 1.4 or greater is
needed to run this program, and it has been tested and operates correctly under
the Windows, Linux and Solaris operating systems.

To start the program, type ``java -jar sightweaver.jar'' in a command prompt or
terminal window. This will start the graphical user interface, which is the
main interface for importing, manipulating and exporting documents.

\section*{Importing Files}

To import an existing document, click on ``Import'' in the ``File'' menu. This
will bring up a standard dialog box, from which the document can be selected.
Supported file types are HTML, XHTML and CSV. The tables must be data tables (as
opposed to layout tables) and must not be nested. If the document does not
contain any tables, the document will not be opened.

Existing table accessibility information in the table will be maintained, and
the program will attempt to guess other informations such as incorrectly
marked up headers, and header associations.

Other document content will also be preserved but will not appear in the 
table window.

\section*{The Table Display}

The first table in the document will be displayed in the table display window.
Other tables can be selected from the ``Table'' menu.

The table is displayed in its \emph{logical} form, so no formatting or styling
information will be displayed. Also, text will be cropped so as to appear in
the table without resizing the cells.

Table \emph{headers} are displayed in a bold font and cells that are
\emph{spanned} are blank with with a diagonal line. 

Cells may be selected by clicking the cell with the left button of the mouse. 
Multiple continuous cells may be selected by clicking again on the next
cell while depressing the ``CTRL'' button.

The following information about selected cells is displayed in the status bar of
the program window:

\begin{itemize}

\item The unique \strong{ID} of the header cell.

\item The \strong{Abbreviation} of the header cell.

\item The associated \strong{Headers} of the header or data cells.

\item The \strong{Axis} of the header or data cells.

\end{itemize}

\section*{Repairing Tables}

The table summary describes the relationship among cells, including their
headers, spanning information or other relationships that may not be obvious
from analysing the structure of the table but that may be apparent in a visual
rendering of the table. The table summary may be edited using the ``Edit Summary''
menu item of the ``Table'' menu.

The table caption is used to describe the table in two to three sentences and may
be edited using the ``Edit Caption'' menu item of the ``Table'' menu.

If the program has not correctly identified the table headers, then these can be
set using the ``Make Header Cell'' menu item. If the program has incorrectly
set the table headers, then use the ``Make Data Cell'' menu item to change
these to data cells.

Cells should be associated with headers in order to identify the relationship
between header and data cells. This can be achieved by selecting the data cells
associated with a header, clicking ``Add Header'' and selecting the correct
header. If all cells under a header can be said to be associated with it, then
this header has a `column' scope. This can be set with the ``Add Header Scope''
menu item. This dialog can also be used to set row scope, which means all the
cells to the right of a header are associated with it.

The table ID is a unique identifier used by cells to refer to their headings
and the header abbreviation should be used to give a terse abbreviation for
headers with long descriptions. These can be set using ``Edit Header Info''.

The cell axis is used to label cells based on some list of categories. Use the
``Edit Axis'' menu item to set this.

\section*{Exporting Files}

Once the table is satisfactory, it may be exported using the ``Export'' menu item
of the ``File'' menu. This displays a standard `Save As' dialog box, from which
the file can be saved as usual. The document will be saved in XHTML Strict 1.0
format, which should be backwards compatible with most browsers.

At this stage, the document will be checked for accessibility. If an error occurs,
a descriptive dialog will be displayed and the document will not be exported.

At any stage during the table repair process, the document HTML source can be viewed
using the ``View Source'' menu item of the ``File'' menu.


