% $Id: background.tex 269 2003-04-22 14:45:04Z mackers $

\begin{quotation}
The power of the Web is in its universality. Access by everyone regardless of
disability is an essential aspect.
\end{quotation}
 -- Tim Berners-Lee, W3C Director and inventor of the World Wide Web

\section{Motivation}

Why Bother With Accessibility?

There are many reasons why most web developers don't concern themselves with
accessibility. Some of these reasons are perfectly valid, but most are
unfounded.

Most believe that visitors with disabilities represent such as small number of
total visitors as to render them insignificant. This perception is hard to
justify when one looks at the figures. As stated in the previous chapter, between 10\% and 20\%
of Americans have some type of disability. A Harris Poll released in
2000 showed that 43\% of these use the Internet, less than non-disabled people,
but considerable all the same. 

The Survey on Income and Program Participation (SIPP, 1999, carried out by the
U.S. Department of Commerce, Economics and Statistics Administration, National
Telecommunications and Information Administration) estimates that 56.7\% of
non-disabled people have Internet access. From these figures it can be derived
that between 2.4\% and 4.8\% of Internet users in the United States have some
degree of disability. 
%\todo{nice graph here?}

To put this in context, consider a popular website with a million unique
visitors a month. By not regarding accessibility, the website owners are
effectively denying full access to up to 48,000 people. 

In the context of this document, it is desirable to isolate people in this
category with some sort of visual impairment. The same Survey of Income and
Program Participation found that 3.5\% of Americans have vision problems, and 
21\% of these people have Internet access. That's about 1,542,410 people
who are not going to able to correctly see a website the way the designers
would. The numbers doesn't seem so abstract when a raw human figure is
put on it.

Another reason why accessibility is not implemented is the myth that it is
expensive. This only really holds true if it is done after the fact. Building
basic accessibility into the design in the first place is effectively free,
requiring developers only to add small amounts of extra code here and there, a
task which should be done without a second thought. In fact, in the long
term, it will probably save the company money in disability lawsuits and
other costs.

For instance, in Australia in June 1999, Bruce Maguire lodged a complaint with
the Human Rights \& Equal Opportunity Commission (HREOC) under a law called the
Disability Discrimination Act. His complaint concerned the Website of the
Sydney Organising Committee for the Olympic Games (SOCOG), which Maguire
alleged was inaccessible to him as a blind person. Maguire successfully won the
case, and the SOCOG was fined \$20,000 in Australian dollars. It was argued
that building accessibility into the project afterwards would have cost A\$2.8,
but experts believe that incorporating it from the start would have only added
2\% to the cost.

Companies must already provide for minority groups. Religious holidays and customs
must be honoured and company buildings must be properly accessible, so
why not websites?

Additionally, a conscious effort is made by website developers to make their
website `accessible' to users of different browsers. So, in that respect, they
must recognise the diversity on which the web is founded, and so should have no
problem envisioning alternative versions of their website. The problem may in
fact be a very basic human fear; ``To imagine your site as experienced by a
blind person is to imagine you are blind yourself.''\cite{joeclark}

The bottom line, however, is that a level of social responsibility must be
applied by website designers and content developers. Although technically
impossible, if a website were to discriminate between different races or
against women, then there would be uproar, so why should discrimination against
people with disabilities be different? 

\section{Legal Requirements}

In most developed countries today there is anti-discrimination legislation
which forbids discrimination or unequal treatment on the basis of disability.
While in most cases this legislation predates the popular uptake of the web, in
general it can be applied to the web (see Olympics example above). Furthermore,
in certain jurisdictions, there is now an explicit requirements for government
departments' and agencies' websites to be accessible. Two such examples are
touched on below.

\subsection{Section 508}

\begin{quotation}

Section 508 requires that Federal agencies' electronic and information
technology is accessible to people with disabilities. The Centre for
Information Technology Accommodation (CITA), in the U.S. General Services
Administration's Office of Government-wide Policy, has been charged with the
task of educating Federal employees and building the infrastructure necessary
to support Section 508 implementation.\cite{section508}

\end{quotation}

The guidelines themselves are similar to W3C's Web Content Accessibility
Guidelines but obviously with a more legal slant. Within the last year Section
508 has come into law, requiring all government websites (with a few minor
exceptions) to comply. 

Additionally, any private sector companies that want to sell to the United
States Government would presumably adhere to these standards. After that,
other private companies might start to take notice.

\subsection{Irish Government Web Publication Guidelines}

In 1999, the Department of the Taoiseach launched Web Publication Guidelines
for Public Sector bodies\cite{irishgov}, with the target for all Government
Department web sites to achieve level AA compliance (see section on Web Content
Accessibility Guidelines) by the end of 2001. A 2002 Web Accessibility in
Ireland Study\cite{warp2002} showed that 100\% of public service websites
failed to meet the WCAG AA accessibility level, proving that a lot of work is
needed.

The guidelines themselves have been compiled by the National Disability
Authority\cite{nda:accessit}, and are generally aligned with W3C's WCAG
guidelines.

\section{Auxiliary Benefits}

Apart from the social and legal liabilities that accessibility negates, they are
a number of auxiliary benefits which may not be entirely obvious, both for the
website owner and for disabled and non-disabled visitors.

Apart from the actual increase in overall visitor numbers by enabling access to
people with disabilities, you may also attract visitors that may not actually
have come in the first place. For instance, blind shoppers may prefer the
comfort and ease of shopping online rather than face the difficulty of getting
in town and navigating real-world shops. Similarly, accessible websites for
hotels allow disabled users to research travel plans and any special
requirements they may need. In effect, these websites can make business gains,
that otherwise wouldn't have existed.

In practice, making a website accessible can have all sorts of other benefits. 
Usually, this very act ensures that the website is of a better quality than its
inaccessible rivals. In general, accessible websites afford a greater level
of standards compliance and logical emphasis, a trait which is picked up by
search engines which look for structural headings to rank a site and index
textual equivalents of images and multimedia, thus boosting targeted traffic
to your website.

The use of image descriptions and the separation of content and presentation
via CSS\footnote{Cascading Style Sheets allow the website's style to be placed
in a separate file which only needs to be downloaded once for the whole site.}
leads to a reduced download time, which is beneficial both to the server's
bandwidth use and to visitors with slow connections. It also means that, should
they wish, users may safely turn images off or easily view the website in
alternative browsers, such as mobile phones and kiosks.

\section{How People with Visual Impairments Use the Web}

Visually disabled users range from from the colour blind to the fully blind.
Table accessibility does not adversely affect the colour blind, who are still
able to see the table structure. Similarly, people with a relatively modest
visual impairments may use a screen magnifier to blow up the size of text,
images, tables and everything else on the screen. They are essentially seeing
the same table structure as a non-visually impaired person, and do not need to
make use of any extra information about the table structure.

Visitors that are totally blind, i.e. don't actually use the computer monitor
use something called a \emph{screen reader}, which is a program that reads or
speaks aloud a description of the website consisting of, but not limited to,
the textual content, descriptions of images and multimedia, `meta' information,
links, and table structural information. While most screen readers are robust
enough to handle even the most badly-formed website, they rely on good
authoring to give the user extra clues as to the exact make-up, structure and
meaning of the website.

A small majority of visually impaired users use braille display - a tablet
consisting of nylon or metal pins controlled by software to give tactual
feedback to the user. Braille software also uses the accessible content and
structure to convey the website to the user.

\section{Accessible Web Design}

\begin{quotation}

The World Wide Web Consortium's (W3C) commitment to lead the Web to its full
potential includes promoting a high degree of usability for people with
disabilities.

WAI, in coordination with organisations around the world, pursues accessibility
of the Web through five primary areas of work: technology, guidelines, tools,
education and outreach, and research and development. \cite{w3c:wcag}

\end{quotation}

Since 1999, the W3C has been working on its Web Accessibility Initiative (WAI).
One of the primary purposes of the WAI is to set out guidelines for
accessibility in web content, authoring tools and in user agents. For the
purposes of this document, the recommendations on web content and authoring
authoring tools are interesting. 

The WAI's Web Content Accessibility Guidelines ``explain how to make Web content
accessible to people with disabilities. The guidelines are intended for all Web
content developers (page authors and site designers) and for developers of
authoring tools.''\cite{w3c:wcag} 

From a technical point of view, these guidelines appear very abstract and so
two more recommendations are provided. The first is \emph{Techniques for Web
Content Accessibility}\cite{w3c:wcagtechs}, which reiterates the above document
from a technical perspective while providing points to the relevant sections in
the more detailed guidelines on individual web technologies. The second set of
recommendations, \emph{HTML Techniques for Web Content Accessibility
Guidelines}\cite{w3c:wcaghtmltechs} is a detailed technical document on how to
implement the above accessibility guidelines in the actual HTML or web content. 

By closely following the guidelines in these documents, web authors can develop
websites that are accessible to people with all sorts of disabilities.
Developers of web site authoring tools, including accessibility validators and
repair tools should use the guidelines to ensure that their programs only
output accessible web content.

\subsection{Web Content Accessibility Guidelines}

The Web Content Accessibility Guidelines cover most elements of a typical
website. Below are described some of the salient points, with the exception
of tables, which are described in more detail in the next section.

Each checkpoint has one of three \emph{priority} levels attached.  Level one
priority checkpoints are basic requirements for accessible websites and
\emph{must} be implemented by content developers. Addressing level two
checkpoints will remove significant barriers to accessing web documents, and
\emph{should} be implemented by content developers. Priority three checkpoints
\emph{may} be satisfied by websites as they will further improve web
accessibility. Websites conforming to these standards can be said to have a
\emph{conformance level} of A, AA or AAA respectively.

The following checkpoints have not been labelled with their
recommended priorities, as the table repair tool will strive to create level
three priority HTML. 

The first guideline is to provide equivalent alternatives to auditory and
visual content. This means that all images, pre-recorded audio and video should
have a text equivalent to allow access both to visually impaired visitors and
to those with reading or cognitive disabilities. For instance, in HTML, every
image should contain an ``alt'' attribute containing a textual description of
the image. Guideline 6 is related to this and requires that new technologies
such as CSS, scripts, applets and frames do not render the page inaccessible,
and that there should be alternative content, if appropriate. Furthermore,
Guideline 8 states that any embedded content that has its own interface
should also be fully accessible.

The second guideline states that colour shouldn't be relied upon to convey
information. For example, links should not be styled using colour alone, 
as users with colour-blindness may not be able to distinguish linked text from
non-linked text.

Guideline 3 requires that markup and style sheets are used correctly. This
guideline encompasses a range of checkpoints including ensuring that the HTML
is well-formed and validates to a standard grammar\footnote{HTML grammars are
sets of standards, e.g. HTML 4.01 to which websites state their adherence by
specifying a document type declaration.}, and that headers, lists, quotations
and tables are marked up in HTML using their intended tags. This ensures
alternative browsing devices can intelligibly understand the organisation of a
page. This guideline also encourages use of CSS style-sheets to store the
style and presentation of a website, freeing other agents from having to
decipher in-line presentation tags.

Natural language use should be identified by declaring the primary language for
the document in general, as well as any other languages used. This allows
speech synthesisers to choose how the text is pronounced. It also allows search
engines to find key words in the particular language used. In practise, content
developers should use a ``lang'' attribute on the HTML element and any other
elements as necessary. Additionally, abbreviations and acronyms should have
their full meaning described using a ``abbr'' or ``acronym'' tag.

Guideline 5 is concerned with ensuring tables transform gracefully and is 
covered in detail in the next section.

The remainder of the guidelines are more abstract, stating that websites are
designed without any specific device in mind (e.g. mice), remain compatible
with known assistive technology shortfalls, and provide context, orientation
and navigation information. 

Naturally, it is also recommended that websites adhere to the other W3C
guidelines and technologies.

\subsection{Accessible Tables}

Guideline 5 is especially interesting in the context of this report as it
encourages the correct markup of tables so that they can be correctly
``transformed'' by accessible browsers and other assistive technologies.

``Tables should be used to mark up truly tabular information (``data
tables'')''\cite{w3c:wcag}, meaning that tables used to layout web
pages should be avoided, as these may cause problems for users of
screen readers.

The following checkpoints are also recommended:

%\todo{sample HTML table containing all these checkpoints and reference in each section to the table}

\subsubsection{Provide Summary Information}

All tables should have summaries. In HTML, these can take the form of a
``caption'' element to describe table in two or three sentences, e.g. ``Number
of civilians killed in the war''. A caption may not always be necessary.

A summary of the table should be provided via the ``summary'' attribute. The
purpose of the this is to describe the relationship among cells, including
their headers, spanning information or ``other relationships that may not be
obvious from analysing the structure of the table but that may be apparent in a
visual rendering of the table''\cite{w3c:wcag}. The summary may also be used to
describe the context of the table in terms of the entire document. An example:
``This table charts the number of civilians killed per day in the war and to
which side they belonged. The first row lists the three nationalities involved
in the war and the first column lists the dates in which they were killed
ranging from 17th March to 21st April''.

\subsubsection{Specify Table Headers}

The table's logical headers must be marked up as headers, i.e. in HTML, data
cells should use ``td'' tags and header cells should use ``th'' cells. 

Repeated rows of headers should be placed in a ``thead'' element and repeated
rows of footers (merely headers at the bottom of the table) should be placed
in a ``tfoot'' element.

\subsubsection{Specify Header Cell Associations}

Table elements should be labelled with appropriate markup to identify the
relationship between data and header cells. Each data cells should have one or
more related headers to which the data in that cell pertains. A ``scope''
attribute can be applied to a table header to imply that this header is
authoritative for cells below it (``col'' scope) or to the left of it (``row''
scope). For more complex tables, data cells can also be labelled with the
``headers'' attribute in order to indicate a relationship with one or more
individual headers.

A third attribute, ``axis'' may be used to label cells so that future browsers
and agents will be able to select data from a table by filtering on 
categories.

\subsubsection{Provide Header Label Abbreviations}

For headers with long descriptions, an ``abbr'' attribute should be used to
give a terse abbreviation so that future screen reading browsers which can read
row and column headers for each cell can cut down on reading time and
repetition.

\subsection{Authoring Tool Guidelines}

The W3C also provides guidelines for developers creating \emph{authoring
tools}\cite{w3c:atag}. An authoring tool can be defined as a program used
to create web content, including:

\begin{itemize}

\item ``WYSIWYG''\footnote{What You See Is What You Get} HTML, XML and CSS
editors.

\item Tools that have the option as saving in a web format (i.e. word 
processors or desktop publishing content)\footnote{It should be noted at
this stage that the most popular word processor, Word, is in breach of 
almost every Authoring Tool Guideline.} and third party tools that
convert these formats to web formats.

\item Tools that produce multimedia where it is intended for use on
the web.

\item Tools for site management for site publication, including tools
that automatically generate web content from databases, and tools
that convert from one format to the other.

\end{itemize}

Since most web content is created using an authoring tool of some description,
they play a critical role in ensuring the accessibility of the web. To this
end, ``authoring tool developers must take steps such as ensuring conformance
to accessibility standards (e.g., HTML 4), checking and correcting accessibility
problems, prompting, and providing appropriate documentation and help.''

A summary of W3C's authoring tool guidelines follows:

\begin{enumerate}

\item Authoring tools should support accessible authoring practices. That is,
authoring tools should generate accessible content and not introduce any
non-accessible content. If the tool imports, transforms or converts existing
content, it should preserve any existing accessibility information.

\item Authoring tools should generate well-formed, valid markup using 
existing W3C standards.

\item The creation of accessible content should be supported. I.e. the author
should be allowed to input text equivalents, captions, auditory descriptions,
etc. that ensure that the final version is accessible. Where mandatory
information is needed, the author should be prompted.

\item Ways of checking and correcting inaccessible content should be provided.
The user should be informed of inaccessible content and assisted in correcting
these problems. Any markup not recognised by the tool should be preserved.

\item The tool's accessibility features should be part of the program's overall
``look and feel'' so that the author easily accepts these features as part
of the operation of the program.

\item Accessibility should be promoted in the help and documentation.

\item The tool itself must be accessible to authors with disabilities using
existing operating system accessibility standards and conventions. 
%\todo{see appendix b?}

\end{enumerate}

\section{Related Work}

\label{relatedwork}

Below is a short list of validators and tools to be used by web content
developers to make their websites more accessible. The list is limited to the
most feature-rich and effective validators and tools that specialise in
table accessibility.

For each tool a brief description is given as well as its features and
limitations, if any. By showing the current state-of-the-art and the limits
there of, this section hopes to justify this report's objective of creating a
tool which will overcome some of these difficulties.

\subsubsection{Bobby}

Bobby is a web accessibility validator ``designed to help expose and repair
barriers to accessibility and encourage compliance with existing accessibility
guidelines.'' It comes in a desktop or an online form and can check single
web-pages or whole sites against WCAG or Section 508 accessibility guidelines.

More information: http://bobby.watchfire.com/

\subsubsection{The Wave}

The WAVE is a free online tool that ``facilitates human judgement in the
accessible design process.'' For a given website, it will `flag' all elements
in the web page and indicate possible problems using different icons.

More information: http://www.wave.webaim.org/

\subsubsection{W3C HTML Validation Service}

While not strictly an accessibility validator, W3C's HTML Validation Service
checks HTML and XHTML web pages for conformance to W3C standards. It is
useful for ensuring that web pages are well-formed and do not contain
any invalid markup.

More information: http://validator.w3.org/

\subsubsection{Tablin}

Tablin is a program that can linearise HTML tables. It displays a textual
version of a table similar to what a screen reader may produce. 

The tool is useful for testing tables to make sure they make sense when
linearised, but does not seem to take in account the full suite of accessible
features, such as headers associations, and so may not be entirely useful for
the purposes of this document.

More information: http://www.w3.org/WAI/References/Tablin/

\subsubsection{Accessify.com's Accessible Table Builder}

This tool lets users create fully accessible tables from scratch. The user
may specify the cell's structure including height, width, headers etc. as well
as the table summary, caption and the data itself and the tool will return
fully accessible table HTML.

The main problem with this tool is that is does not allow the import of
existing tables/documents, requiring tedious cell-by-cell input. It also does
not permit complex header hierarchies, cell axes or header abbreviations.

More information:\newline
http://www.accessify.com/tools-and-wizards/accessible-table-builder\_step1.asp

\subsubsection{A-Prompt}

A-Prompt (or Accessibility-Prompt) is a free software tool designed ``to improve
the usability of HTML documents by evaluating Web pages for accessibility
barriers and then providing developers with a fast and easy way to make the
necessary repairs.''

The program itself is quite sophisticated, allowing the user to check web pages
against all W3C and Section 508 conformance levels. The author has full control
over accessibility features such as text equivalents, form accessibility and
table accessibility.

Its table support is reasonable, allowing the user to enter summary and caption
information for each table and abbreviations for each header. However it only
recognises headers in the first row or first column, so more complex tables
are not supported. Also, while it recognises the absence of header scopes
and associations, it does not allow in-line editing and suggests that the
author add these manually.

More information: http://aprompt.snow.utoronto.ca/

%\subsubsection{WebAIM.org's Screen Reader Simulation}

%More information: http://www.webaim.org/simulations/screenreader

