% $Id: implementation.tex 269 2003-04-22 14:45:04Z mackers $

\section{Road-map}

The project road-map will roughly adhere to the following sequence:

\begin{enumerate}

\item The project will be begin with the testing and evaluation of the external
tools, components and libraries chosen to complement the table repair tool.
This will reveal at an early stage if the tools are suitable; that is, do they
integrate into the program satisfactorily, and do the results meet what is
required of them. If they do not meet the requirements then other tools must be
investigated or the requirements altered. A short description of each external
component follows:

\begin{itemize}

\item Probably the most important external library is JTidy, used for converting
HTML documents to XHTML. JTidy must be tested for robustness in converting generic
HTML and whether it is compatible with the ``Pre-Parser'' concept. JTidy must not
remove existing accessibility information from the document.

\item The CSV to XHTML convertor should be tested to determine if it is suitable
and robust enough to use in the project.

\item Xerces is a java XML parser that is considered to be more faster and more
sophisticated than the built-in java XML parser. This library should be tested
for compatibility with the standard java DOM classes.

\item Although Swing is built in to java, it does not necessarily have to be
used. It should be tested to see if it is suitable for this project, particularly
the table display component and its accessibility features.

\item CVS\footnote{Concurrent Versions System -- http://www.cvshome.org/} should 
be used to maintain version information and allow for easy backup.

\end{itemize}

In addition, Java's localisation and javadoc features will be examined to see
if they can be used with this particular project.

\item To begin the actual coding stage, a ``bare-bones'' version of a single
document importer will be implemented.  The XHTMLImporter is the most suitable
choice, as it takes the least effort to import (i.e. using the XML parser). At
this stage, the importer will be able to read the document and extract the most
basic information such as normal (non-spanned) cells, and the table summary and
caption. Some sample XHTML documents are created at this stage.

\item Now that an internal representation of a table is available, a basic
graphical user interface should be developed to display the basic table, in
order to ensure that the internal representation is consistent with the
actual table in the document. It is important to have a graphical representation
at this stage as it would be difficult or impossible to view the overall table using
traditional debugging methods.

\item Once the basic table model is implemented and working, further features
can be added to the table backend:

\begin{itemize}

\item Cells in the table should be checked for cell and row spanning and handled
according to W3C HTML recommendations\cite{w3c:xhtml}. 

\item If the document is missing headers, use the header finder algorithm as described
in the requirements.

\item The importer must take into account existing accessibility information in 
the table. Scopes, headers associations, axes and abbreviations must be preserved
and stored in the table or cell objects.

\item If the table accessibility information for table header associations is missing
or incomplete, find this using the heuristic algorithms described in the 
requirements.

\end{itemize}

\item The user interface should be updated to use the new features from the previous
step:

\begin{itemize}

\item The user interface will take spanned cells into account when rendering
the table, either by spanning them as in browsers, or rendering spanned cells
distinctly.

\item The table status bar is implemented to show the attributes of the currently
selected cell(s). This also requires the cell selection mechanism to be implemented.

\end{itemize}

\item After these basic requirements have been implemented, the rest of the graphical
user interface should be added so that subsequent features can be easily tested:

\begin{itemize}

% -- moved to problems -- \item In the user interface window large tables should scroll.

\item Table switching provided via the table menu.

\item The application title bar should consist of application title, document title and table title.

\item Table menus and menu-items should be enabled or disabled by context of open/closed document
and selected cells as described in requirements.

\item The interface and actual functionality of the dialogs.

% -- moved to problems -- \item File filters for the ``Import'' dialog, e.g. ``*.html''.

% -- moved to problems -- \item The mouse cursor should change to the ``Hourglass'' during imports, as they may take
%a short while in which the application would appear to be doing nothing.

\end{itemize}

\item The HTML importer is now ready to be implemented. This should be a
trivial matter, as JTidy has already been tested, and the XHTML importer has
been implemented. In theory, JTidy will open the HTML and give back XHTML,
which can be used by the XHTML importer. Pre-parser support should also be
implemented at this stage. 

\item The CSV importer will also be implemented (using the CSV to XHTML convertor).

\item The XHTML exporter will now be implemented. This should be relatively
trivial as the document's tables and cells have been keeping a record of their
DOM structure to reflect each automated or user change. This means that a
current DOM document is available to export. It might be desirable to use
JTidy's ``pretty-printer'' to output aesthetic XHTML (i.e. indented, etc.).
The ``View Source'' feature, which uses the exporter, should be implemented at
this stage.

\item Now that all the main requirements of the tool have been implemented, the
usability can be focused upon. This includes providing meaningful feedback for
errors when importing and exporting documents. The dialog descriptions are also
entered (if not already).

\item Finally, any remaining Swing accessibility features are added. This includes adding
mnemonics to menus and buttons, adding descriptions to menu items and menus and ensuring
all labels are correctly associated with a menu item.

\item Any remaining javadoc documentation should be added now.

\end{enumerate}

\section{Samples}

The sample set consists of a series of files that will be used for testing
the program's import function as to whether it meets the requirements. The
files will also expose any bugs in the software. 

The criteria for the sample set is as follows:

\begin{enumerate}

\item Documents outputted from each of the supported programs identified
in the requirements section.

\item Tables containing all the accessibility features that are 
required to be retained during the import process, e.g. scope, headers, axis
and abbreviation.

\item A table with column groups.

\item A table with rows shorter than others.

\item A table with headers marked up as bold.

\item A table with a `thead' element instead of headers.

\item A table with no headers or header associations.

\item A table with cells spanning rows and columns.

\item A document with multiple tables.

\item A document with (non-accessible) text outside of a table.

\item A document with a nested table.

\item A document using a layout table.

\item A document with no tables.

\item At least one sample should be an XHTML file.

\item A Word document in its native format.

\item A sample CSV file.

\item A sample unsupported file.

\item Several mal-formed HTML files to test JTidy (these are not including
in the sample set).

\end{enumerate}

\section{Problems Encountered}

During the development of the program, several problems were encountered.
These, together with their solutions, are documentated below.

\label{problems}

\subsection{Localisation}

\begin{itemize}

\item The first issue that became evident was the problem with hard coding
English-language text in the user interface. If, at some future stage, the
program needed to be \emph{localised}, i.e. translated into another language,
then all user interface text in the program would have to be located and
translated in the source code, requiring a recompilation. This would be a time
consuming and tedious task. Instead, all English-language user interface
strings were placed into a \emph{String-Bundle} file and loaded via the
String-Bundle interface. Should a localised version be developed, the human
translator need only translate those strings in the file and rename the file
appropriately and java will use that file instead.

\item From this it also became clear that menu and buttons mnemonics (i.e. 
shortcut keys) could not be hard-coded into the application, as they would 
be different for every language. For example, in English, the mnemonic
for the ``File'' menu would be `F', whereas in Spanish, the mnemonic for
the same menu (``Archivo'' when localised) would be `A'. To this end, the
menu, menu item and buttons mnemonics were also placed in the localised
String-Bundle file.

\item Long text fields presented a problem during the localisation process.
When the length of the text is known, it is possible to split the text so that
it appears correctly and does not stretch the size of the dialog or become
truncated. However, it is more difficult to control the layout of the text when
it must be loaded from a String-Bundle file. For short texts a backslash-n was placed
in the string to cause the Swing component to insert a new line. For longer
texts it was discovered that Swing's JLabel can render a string as HTML, thus
causing it to \emph{wrap} correctly inside the dialog or interface. As a
result, long texts are now represented as in-line HTML documents in the
String-Bundle file.

\item Also, because the main dialogs are being implemented from scratch, the
text and keys for the default ``OK'' and ``Cancel'' buttons must be localised,
whereas normally a dialog using the standard JOptionPane constructs would
handle this.

\item The French language bundle is included with the program. If the operating
system's locale is set to French, then Java will automatically select this
locale to use, and the dialogs and user interface strings will appear in
French. Without having to resort to changing the whole operating system's
locale, the French language bundle can be tested by invoking the java command
with the ``user.language'' system property set to `fr' and the ``user.region''
system property set to `FR'.

\end{itemize}

\subsection{Importing Tables}

\begin{itemize}

\item The HTML, XHTML and CSV importers have been implemented in this release
of the program but the RDF importer has not. It would be desirable to allow
for this importer to be implemented at a later stage, perhaps by a third person
without any alteration of the existing code. 

This required the program to \emph{test} whether the class existed or not.
Regular java code would create a new instance of the RDFImporter using the
\code{new} keyword in order to use it.  However, this would give a compilation
error as the class doesn't exist. Instead the \code{newInstance()} method of
Java's \code{Class} class is used to create the instance of RDFImporter. If the
class has been implemented then it can be used. Otherwise, an exception can be
caught and the user informed that RDF is not available.

\item The Pre-Parser stage of the HTML importer requires a knowledge of the
program in which the HTML was originally created. The requirements section of
this document stated that this should be taken from the document's `Generator'
meta tag. In practice, however, extracting the value of this attribute poses a few problems. Firstly, the case of
the element, attribute name or attribute value may be lowercase, uppercase or a
mixture of the two. Secondly, quotes around the attribute value may be single
quotes, double quotes or missing altogether. The various supported programs
generate a combination of all these problems. It was concluded that the best
means to extract the exact generator attribute value was via a \emph{regular
expression}. Regular expressions are a powerful way to match patterns in
strings of text. In this case, the regular expression is used to match
case-insensitively with optional quotes around the attribute values.

\item The ``Make Header'' and ``Make Data'' features need to change the tag
name of the cell in the document's DOM. However, there is no DOM function
to change a tag's name. This necessitated a new function to do this, which 
consisted of the cloning of the old element's child nodes and attributes into
a new element with the new name and swapping them in the DOM tree.

\item A seemingly intermittent problem occurred when importing XHTML files.
Upon attempting to load and parse the XHTML file, a
\emph{NoRouteToHostException} was being thrown. After some investigation, it
was realized that this problem was only occurring on certain machines, that is,
those without a direct external connection to the Internet. What was in fact
happening, was that the XML parser was attempting to download the XHTML
DTD\footnote{The DTD file is used to check the XHTML file for \emph{validity}
and has a uniform worldwide location of
http://www.w3.org/tr/xhtml1/Dtd/xhtml1-strict.dtd} and was unable to because it
was unaware of the proxy settings needed. To overcome this problem, java
needed to be started with the system \code{http.proxyHost} and \code{http.proxyPort}
settings defined correctly.

\end{itemize}

\subsection{Graphical User Interface}

\begin{itemize}

\item Selected table cells are indicated by a different background colour
(light blue). This was implemented by returning a new cell component albeit
with a different background colour. However, this only set the background
colour behind the text and not in the whole cell, which looked a bit
unconventional and unintuitive. The cell component was then set to be the
maximum possible size, meaning that it would stretch inside the table cell
and the background colour would include the entire area.

\item Large tables caused a problem with the table display mechanism. If the
number of rows exceeded the size of the application window, then Swing would
reduce the height of all the rows to make room. This resulted in unreadable
text, which was problematic in conveying the header information to the user.
This problem was solved by placing each table in a \emph{ScrollPane} which
provides a vertical scrollbar which does not force Swing to squash the table
rows and can be used by the user to scroll the table up or down to reveal
the rest of the table.

\item Each cell renders the first text node found in the corresponding cell in
the table in the import document. However, if text in the cell is within
another element, for example a paragraph or bold element then this is not
returned by the appropriate DOM function. A new DOM function was written to
return \emph{all} text found in the context of the cell, including that found
in sub-elements. This ensured that the cell's made sense in all contexts.

\item Some importers take a short while to open and parse the document. This is
especially noticeable with the HTMLImporter, which must call JTidy to convert
the import document. During this time, the application does not appear to do
anything, even to ``hang'' momentarily. Depending on the size of the document,
the user may be unaware of what is happening and may believe something has gone
wrong as the program is not responding. To overcome this ambiguity, the mouse
cursor is temporarily changed to the wait or `hourglass' cursor while the
document is being imported. This provides visual feedback to the user who knows
that something is happening in the background and that he/she must wait.

\item The application was developed on a Linux system using the default
Swing application look-and-feel. For experienced Linux users, this interface
poses no problem as they are used to and feel comfortable with a range of
different look-and-feels. However, Windows users may not be entirely 
comfortable with strange interfaces. To this end, the program now uses the
`system' look-and-feel, which causes Swing to render components differently
depending on the operating system under which it is run.

\item Testing the software in Windows revealed that the JTable was rendered
slightly different than during development, in that blank JTable headers 
were displayed. This was fixed.

\end{itemize}

\subsection{Packaging}

\begin{itemize}

\item The program is to be deployed as a JAR archive file, and testing as this
revealed some problems. The first problem was that the ``.properties'' 
String-Bundle files were not correctly loaded, resulting in blank user interface
strings. This problem was solved by adding a package qualifier to the function
loading these files.

\item Also, the default class to run should be specified in the JAR file, as this
simplifies the execution of the program. This involved creating a \emph{Manifest}
file for the JAR package which stated which was the default class to run.

\end{itemize}
