% $Id: conclusion.tex 269 2003-04-22 14:45:04Z mackers $

\section{What Was Achieved}

A lot of new technologies and skills were learnt during the course of the
project. 

As demonstrated in chapter 2, an extensive understanding of the core
technologies involved was undertaken. This included reading W3C's extensive and
detailed specifications and recommendations for HTML and XHTML and
accessibility guidelines for web content and authoring tools.

There were other, secondary technologies that were learnt as part of the
project implementation. These included using Swing to develop the 
graphical user interface and Xerces has an XML and DOM parser. Javadoc
was also employed to create the project API.

Auxiliary achievements include learning and using CSV as a version control
system for the program source code and learning to use \LaTeX\ to write and
typeset this report.

It was also very interesting and beneficial to go through the whole software
engineering process, from researching a topic, defining requirements and
specifications right through to writing the actual software and evaluating it.
There was an incredible sense of satisfaction and achievement when the project
was successfully completed. 

\section{Future Work}

\label{futurework}

The project as a whole can be said to have reached a level
of stability and completeness; a true ``version 1.0''. However,
this does not preclude the possibility of improvements and
other features being added.

Naturally, the first improvement would be to fix the bugs listed
in the ``Known Issues'' section on page \pageref{knownissues}. All
of these should be fairly trivial to fix.

Other improvements that could be made are listed below. They
were not listed in the known issues list because they were not
part of the original requirements, and did not need to be 
implemented.

\begin{itemize}

\item ``Rowgroup'' and ``colgroup'' scopes could be supported when
importing documents. Although the supported output formats do
not use these attributes, supporting them would mean that the
program supports all table accessibility features. (bug 0016)

\item The user should be informed if the program could not
identify or guess any header information. This would ensure
that the user does not attempt to export the document without
this fundamental information. (bug 0020)

\item In the absence of a table summary, either in the original
document or supplied by the user, the program could generate
the summary based on the number of rows and columns and other
structural information in the table. (bug 0021)

\item The table cell renderer is too slow. This is probably
because each time the table is refreshed a new component
is created for the cell and return to the table renderer.
Instead, the same cell component should be reused. (bug 0025)

\item If a generator tag is present in the imported document,
it is currently replaced with a SightWeaver generator tag.
However, if one is not present, then it cannot be replaced. A
new generator tag should be added. (bug 0023)

\end{itemize}

Here are some other features that the program would benefit from:

\begin{itemize}

\item The RDF importer is currently supported, but not implemented. If one was
to implement this, a ``RDFImporter'' class would be created to implement the
``Importer'' interface, in a similar manner to the ``CSVImporter''. The
compiled class can then just be dropped into the sightweaver package and the
Document class will use this for RDF files.

\item Other importers that could be implemented include a native binary Word
format importer (difficult!), a DOCBOOK importer, etc. 

\item It would also be nice if a URL could be provided to import documents, and
if the exporter could publish directly to the web via FTP or WEBDAV, allowing
the user to use the program as a final step before publishing to the website.

\item Cell spanning is currently ugly and unintuitive. Word and web user are 
accustomed to seeing spanned cells appearing in the same cell, as opposed
to having a placeholder cell instead. In this case, it might be possible
to turn off the cell borders for the spanned cells to give the impression
that the spanned cells are a single cell with text that is top- and left-aligned.
Anything more sophisticated than that would require a more drastic approach
involving moving away from the current JTable approach to manually place the
cells. Using this approach, a single cell could be place anywhere in the
table, allowing the impression that the cell was spanned and has centre- and
middle-aligned text.

\item A feature that would be of great help to content developers with an
accessibility conscience would be a ``screen-reader simulation'' mode. This
would involved a dialog that would either display or speak a version of the
table or document as it would appear to a user using a screen-reader. This
would give confidence to the content developer that his or her tables
actually make sense when viewed in this manner.

\item Large organisations may use an existing content development package or
system that is unsuitable for create accessible content. This program would be
attractive in that situation, as it could convert the existing format to an
accessible one. However, most organisations have a large amount of documents
that would have to be converted; a tedious task if they are all similar but
needed to be converted manually. Ideally, this tool would `learn' from a single
document and be able to apply that to the rest in the document set. This solution
would require a fair amount of clever AI techniques.

\end{itemize}

It is also quite practical for the program to be included as part of another,
larger program. All the functionality of the document import and export
algorithms has been completely separated from the graphical user interface.
Also, the main functionality of the user interface has been modularised into
the SWTable classes, meaning that the actual window itself is merely a holder
for the menus (which simply calls functions of the table) and the status bar
(which is manipulated by the table). By implementing the ``SWJTableContainer''
interface, any other window, frame or other container can be fully compatible
with the bulk of the program. It should also be possible to implement the
program as a web service, however some user feedback is always needed so the 
process can never be fully automated.

As the program is fully internationalisable, it's relatively easy to localise
into different languages. Translators need only make a copy of the
``sw.properties'' file and translate each string therein (with a
couple of marked exceptions). The new file is then saved with a new filename
indicating the language and country code. For example, Austrian German
would be saved in a file called ``sw\_de\_AT.properties''.

It is intended to release the program source code under the GPL\cite{gpl} 
license to encourage and allow people to implement some or all of these
improvements as they see fit.

\section{Conclusions}

Overall, this project can be said to have been successful. The motivation
behind the accessibility movement was presented both from a social and legal
perspective. The report then went on to explain the accessibility standards and
legislation that exists today, as well as the problems with websites as they
stand today. Existing technologies and other related work in the fields of
assistive technology and accessibility tools was then explored.

The report then proposed a solution to one aspect of accessibility; tables. The
solution was in the form of a tool to assist authors in creating fully
accessible tables. The requirements and specification of the tool were
detailed, as well as the architecture and design process. There then followed a
discussion of the problems that occurred during development and the outstanding
issues that remain. The tool was then evaluated against accessibility and legal
standards.

In this chapter, some possible future work in this area was discussed, as well
as what was achieved in the project as a whole.

The reader may additionally wish to peruse this report's appendices, which contain
the day-by-day project diary, the tool's user manual and details on the contents
of the accompanying CD-ROM.
